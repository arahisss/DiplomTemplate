\newpage
\begin{center}
  \textbf{\large АННОТАЦИЯ}
\end{center}

Наименование работы: разработка веб-приложения для проектирования базы
данных с помощью DBML с последующей генерацией
SQL кода для различных СУБД.

Цель работы: создание программного обеспечения для проектирования баз данных на основе языка разметки DBML, которое предоставит пользователям удобный инструмент для визуального создания, редактирования и управления схемами баз данных, а также поддержку командной работы, экспорта/импорта SQL-кода и использования готовых шаблонов. Программное обеспечение будет включать десктопное приложение и веб-платформу, что обеспечит гибкость использования как в онлайн, так и в офлайн-режиме.

Задачи для достижения цели: анализ предметной области и существующих решений, проектирование структуры данных и пользовательского интерфейса, реализация функционала, тестирование удобства использования и
разработка стратегии продвижения приложения.

Объект и предмет исследования: процесс разработки программного обеспечения для проектирования базы данных на основке языка разметки DBML для последующей генерации SQL-кода.

Работа состоит из 3 глав, Введения, Заключения. Общий
объем работы составляет . . . страниц, из них . . . приложений на . . . листах.
В работу включены . . . рисунков и . . . таблиц. Библиографический список
состоит из . . . источников, из них . . . печатных источников и . . . интернетисточников.

Во Введении описаны цель и задачи работы, объект и предмет исследования, его актуальность, новизна и практическая значимость.

В Первой главе описывается анализ предметной области, изучение аналогов и конкурентов, а также обоснование выбора технологий и платформ
для разработки.

Во Второй главе представлена разработка структуры данных, проектирование пользовательского интерфейса и реализация основных функций приложения.

В Третьей главе раскрыты технические аспекты внедрения и опытной
эксплуатации полученного практического результата ВКР, включая план юзабилити-тестирования и анализ его результатов.

В Заключении представлены выводы по поставленным задачам и обоснование практической значимости разработанного приложения.

\onehalfspacing
\setcounter{page}{5}

\newpage
\renewcommand{\contentsname}{\centerline{\large СОДЕРЖАНИЕ}}
\tableofcontents

\newpage
\begin{center}
  \textbf{\large ВВЕДЕНИЕ}
\end{center}
\addcontentsline{toc}{chapter}{ВВЕДЕНИЕ}


Современные информационные системы всё чаще требуют эффективного управления большими объёмами данных, что делает проектирование баз данных одной из ключевых задач в разработке программного обеспечения. Однако существующие инструменты для проектирования баз данных зачастую имеют ряд ограничений, таких как недостаточный функционал в бесплатных версиях, отсутствие поддержки командной работы, ограниченные возможности для офлайн-использования и отсутствие готовых шаблонов. Эти проблемы особенно актуальны для студентов, начинающих разработчиков и небольших команд, которые не могут позволить себе дорогостоящие решения.

Актуальность данной выпускной квалификационной работы (ВКР) обусловлена растущей потребностью в удобных и доступных инструментах для проектирования баз данных. Язык разметки DBML (Database Markup Language) становится всё более популярным благодаря своей простоте и удобству для описания структуры баз данных. Однако существующие решения на основе DBML имеют ограниченный функционал. Кроме того, большинство из них представлены только в виде веб-приложений, что ограничивает возможности работы в офлайн-режиме.

Целью работы является разработка программного обеспечения для проектирования баз данных на основе языка разметки DBML, которое предоставит пользователям удобный инструмент для визуального создания, редактирования и управления схемами баз данных. Для достижения этой цели необходимо решить следующие задачи: провести анализ предметной области, изучить существующие аналоги, выбрать подходящие технологии и платформы для разработки, спроектировать структуру данных и пользовательский интерфейс, разработать интерфейс редактора диаграммы, реализовать функции синхронизации между DBML и диаграммой, функции генерации SQL-кода из DBML, а также разработать план тестирования и стратегию продвижения. Объектом исследования являются процессы проектирования баз данных, включая визуализацию схем, генерацию SQL-кода и поддержку командной работы.
Предметом исследования является разработка программного обеспечения для проектирования баз данных на основе DBML, которое представляет собой веб-приложение.

Практическая значимость работы заключается в предоставлении пользователям бесплатного и функционального инструмента для проектирования баз данных, который может быть использован в образовательных целях, для профессиональной разработки и командной работы. Разработанное программное обеспечение позволит ускорить процесс проектирования баз данных, повысить качество разработки и снизить затраты на использование специализированных инструментов.

Целевая аудитория разрабатываемого программного обеспечения включает студентов, начинающих и профессиональных разработчиков, команды разработчиков, стартапы, небольшие компании, аналитиков и архитекторов баз данных, а также фрилансеров. Для каждой из этих групп инструмент предоставит уникальные возможности, такие как визуализация схем баз данных, поддержка командной работы, экспорт и импорт SQL-кода, а также использование готовых шаблонов.

Данная ВКР направлена на решение актуальной проблемы и создание
востребованного продукта. Разработанное приложение, сочетающее функциональность, гибкость и удобство, не только упростит процесс проектирования баз данных, но и сделает его более доступным для широкого круга пользователей.

% \textbf{Цель магистерской квалификационной работы} -- установить связь \\ дальнодействия притяжения потенциала взаимодействия и спектров возбуждений с транспортными свойствами жидкостей, а также выявить влияние дальнодействия притяжения на скорость нуклеации.

% \textbf{Задачи магистерской квалификационной работы:}
% \begin{enumerate}
% \item Расчет фазовых диаграмм для 2D и 3D систем частиц, взаимодействующих посредством обобщенного потенциала Леннарда-Джонса с различными степенями притяжения. 
% \item Адаптация метода кластеризации данных DBSCAN для изучения молекулярных систем и его сравнение с другими методами.
% \item Расчет и анализ транспортных свойств и коллективных возбуждений на жидкостных бинодалях.
% \item Применение нового метода распознавания фаз для изучения скорости нуклеации в переохлажденных системах Леннарда-Джонса с различным дальнодействием притяжения. 
% \end{enumerate}


% \textbf{Научной новизной обладают следующие результаты магистерской
%   квалификационной работы:}
% \begin{enumerate}
% \item Установлено, что подвижность имеет линейную температурную зависимость в широком диапазоне на бинодали жидкость-газ.
% \item При увеличении дальнодействия потенциала увеличивается отношение температур критической к тройной точке.
%   Кроме того, при этом уменьшается наклон температурной зависимости подвижности.
% \item Отклонение подвижности от линейной зависимости при высоких температурах коррелирует с переходом спектров возбуждений от осцилирующего к монотонному виду.
% \end{enumerate}


% \textbf{Апробация} основных результатов магистерской квалификационной работы проводилась на следующих конференциях:
% \begin{enumerate}
% \item XX Школа-конференция молодых ученых <<Проблемы физики твердого тела и высоких давлений>>, Сочи, 16-26 сентября 2021г.
% \item Современные тенденции развития функциональных материалов, Сочи, 11-14 ноября 2021г.
% \item Dynamic phenomena workshop 2022.
% \end{enumerate}


% \textbf{Публикации:}
% \begin{enumerate}
% \item Kryuchkov, N. P., Dmitryuk, N. A., Li, W., Ovcharov, P. V., Han, Y., Sapelkin, A. V., and Yurchenko, S. O. (2021). \\ Mean-field model of melting in superheated crystals based on a single \\ experimentally measurable order parameter. Scientific reports, 11(1), 1-15.
% \item Yakovlev, E. V., Kryuchkov, N. P., Korsakova, S. A., Dmitryuk, N. A., Ovcharov, P. V., Andronic, M. M., ... and Yurchenko, S. O. (2022). 2D colloids in rotating electric fields: A laboratory of strong tunable three-body interactions. Journal of Colloid and Interface Science, 608, 564-574.
% \item Tsiok, E. N., Fomin, Y. D., Gaiduk, E. A., Tareyeva, E. E., Ryzhov, V. N., Libet, P. A., ... Yurchenko, S. O. (2022). The role of attraction in the phase diagrams and melting scenarios of generalized 2D Lennard-Jones systems. The Journal of Chemical Physics, 156(11), 114703.
% \end{enumerate}

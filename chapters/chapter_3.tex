\newpage
\begin{center}
  \textbf{3 ФУНКЦИОНАЛЬНОЕ ТЕСТИРОВАНИЕ И ОЦЕНКА РЕЗУЛЬТАТОВ}
\end{center}
\addcontentsline{toc}{section}{3 ФУНКЦИОНАЛЬНОЕ ТЕСТИРОВАНИЕ И ОЦЕНКА}

\textbf{3.1 Разработка плана функционального тестирования}
\addcontentsline{toc}{subsection}{3.1 Разработка плана функционального тестирования}


План функционального тестирования разрабатывается для
систематической проверки всех функций приложения и выявления возможных
ошибок на ранних этапах. Такой план включает перечень тестируемых
функций, сценарии использования, ожидаемые результаты и критерии
успешности.

Цель тестирования: убедиться, что все основные функции приложения
реализованы корректно, работают стабильно и соответствуют требованиям
пользователя.

Область тестирования: покрываются все ключевые модули: регистрация
и авторизация, выход из системы, создание проекта, удаление проекта, добавление в избранное, добавление других участников в проект, работа dbml редактора, работа интерфейса для создания диаграмм, сохранение проекта, экспорт SQL-дампа.

Основные тестовые сценарии:

\begin{enumerate}
\item Регистрация пользователя:
    \begin{enumerate}
     \item  Ввод валидных данных и успешная регистрация.
     
     \item Попытка регистрации с уже существующими email или логином.
     
     \item Ввод некорректных или неполных данных (пустые поля, разные
пароли).
    \end{enumerate}

\item Авторизация пользователя:
    \begin{enumerate}
         \item Вход с корректными данными.
         \item Вход с неверным логином или паролем.
         \item Проверка перехода между окнами авторизации и регистрации.
    \end{enumerate}

\item Выход из системы:
    \begin{enumerate}
        \item Проверка работы кнопки "Выйти" для всех пользователей.
        \item Очистка полей ввода при возврате к авторизации.
    \end{enumerate}

\item Создание проекта:
    \begin{enumerate}
        \item Проверка создания проекта при вводе всех данных.
        \item Проверка создания проекта при отстутвии даннных в одном из полей.
    \end{enumerate}

 \item Удаление проекта:
    \begin{enumerate}
        \item Проверка работы контекстного меню проекта.
        \item Проверка удаления проекта из базы данных после нажатия на кнопку удаления.
    \end{enumerate}

 \item Добавление проекта в избранное:
    \begin{enumerate}
        \item Проверка появления проекта в таблице favourite\_project после нажатия на иконку избранного.
        \item Проверка отображения проекта в разделе "Избранное" в меню.
    \end{enumerate}


 \item Добавление других участников в проект:
    \begin{enumerate}
        \item Проверка добавления существующего участника в проект.
        \item Проверка базы данных project\_access.
        \item Вход в систему под аккаунтом добавленного участника.
        \item Проверка отображения проекта в разделе "Мои проекты" и "Общие проекты" в меню.
        \item Проверка возможности внесения изменений в проект.
    \end{enumerate}

 \item Работа dbml редактора:
    \begin{enumerate}
        \item Открыть редактор проекта.
        \item Внести в редактор dbml тестовый код.
        \item Сохранить проект и обновить страницу.
        \item Проверка отображения внесенного кода.
        \item Проверка подсветки синтаксиса.
        \item Проверка работы валидации.
    \end{enumerate}

 \item Работа интерфейса для создания диаграммы:
    \begin{enumerate}
        \item Открыть редактор проекта.
        \item Нажать на кнопку "Добавить узел" два раза.
        \item Задать название таблицам, добавить несколько полей, установить типы полей.
        \item Провести стрелку связи между двумя полями.
        \item Обновить страницу.
        \item Проверка отображения диаграммы в том же виде и на тех же местах.
    \end{enumerate}

 \item Сохранение проекта:
    \begin{enumerate}
        \item Открыть редактор проекта.
        \item Нажать на кнопку "Добавить узел".
        \item Обновить страницу.
        \item Проверка отображения нового узла и соответствующего dbml кода.
        \item Нажать на кнопку "Добавить узел", а затем на кнопку сохранения.
        \item Обновить страницу и проверить отобржение нового узла и соответствующего dbml кода.
    \end{enumerate}

 \item Экспорт SQL-дампа:
    \begin{enumerate}
        \item Открыть редактор проекта.
        \item Вставить готовый тестовый dbml код.
        \item Нажать на кнопку экспорта.
        \item Проверить, скачался ли sql-файл.
        \item Открыть файл и проверить его на корректность.
    \end{enumerate}
    
\end{enumerate}


\textbf{3.2 Проведение тестирования}
\addcontentsline{toc}{subsection}{3.2 Проведение тестирования}

Тестирование проводилось вручную на операционной системе Windows 10, с использованием различных наборов тестовых данных (валидных, некорректных и пограничных случаев). Все основные пользовательские сценарии были воспроизведены последовательно, с фиксацией результатов и выявленных проблем. Для тестирования использовались браузеры Chrome (версия 112.0), Firefox (версия 111.0) и Edge (версия 112.0).


\begin{enumerate}
\item Регистрация пользователя:
    \begin{enumerate}
     \item Сценарий: ввод валидных данных (уникальный логин, корректный email, совпадающие пароли).
     
     \item Результат: регистрация проходит успешно, происходит автоматическая авторизация и переход на главную страницу.
     
     \item Сценарий: попытка регистрации с уже существующим email.

     \item Результат: система выводит сообщение "Этот email уже используется", поля паролей очищаются.

     \item Сценарий: ввод пустых полей.

     \item Результат: система выводит сообщение "Заполните это поле" при вводе пустых полей.

     \item Вывод: механизм регистрации работает корректно, все ошибки валидации обрабатываются должным образом.
    \end{enumerate}

\item Авторизация пользователя:
    \begin{enumerate}
     \item Сценарий: вход с корректными учетными данными.
     
     \item Результат: успешный переход на главную, в заголовке отображается кнопка "Выйти".
     
     \item Сценарий: вход с неверным паролем.

     \item Результат: сообщение "Неверный email или пароль".

     \item Сценарий: переход между формами авторизации и регистрации.

     \item Результат: переход осуществляется мгновенно, все ранее введенные данные в полях очищаются.

     \item Вывод: система авторизации обеспечивает надежную защиту учетных записей.
    \end{enumerate}

\item Выход из системы:
    \begin{enumerate}
     \item Сценарий: нажатие кнопки "Выйти" из личного кабинета.
     
     \item Результат: сессия завершается, происходит переход на страницу авторизации, токен доступа аннулируется.
     
     \item Сценарий: повторная авторизация после выхода.

     \item Результат: требуется полный ввод учетных данных, предыдущая сессия не восстанавливается.

     \item Вывод: механизм выхода реализован безопасно, данные сессии полностью очищаются.

    \end{enumerate}


\item Создание проекта:
    \begin{enumerate}
     \item Сценарий: создание проекта с указанием всех обязательных полей (название, тип БД).
     
     \item Результат: проект успешно создается, появляется в списке "Мои проекты".
     
     \item Сценарий: попытка создания без указания названия.

     \item Результат: кнопка "Создать" неактивна, поле названия выделяется с подсказкой "Заполните это поле".

     \item Вывод: валидация при создании проекта работает корректно.

    \end{enumerate}
    
\item Удаление  проекта:
    \begin{enumerate}
     \item Сценарий: клик на контекстное меню для удаления проекта.
     
     \item Результат: контекстное меню успешно отображается.
     
     \item Сценарий: попытка клика на кнопку удаления проекта.

     \item Результат: проект удаляется из базы данных и из списка проектов.

     \item Вывод: удаление проекта работает корректно.

    \end{enumerate}

\item Добавление проекта в избранное:
    \begin{enumerate}
     \item Сценарий: клик кнопку для добавления проекта в избранное.
     
     \item Результат: проект отображается в разделе "Избранное" и добавился в базу данных favourite\_projects.

     \item Вывод: добавление проекта в избранное работает корректно.

    \end{enumerate}

\item Добавление других участников в проект:
    \begin{enumerate}
     \item Сценарий: заполнение формы добавления участника корректным email.
     
     \item Результат: после авторизации под аккаунтов добавленного участника проект отображается в разделах "Мои проекты" и "Общие проекты". Есть возможность внесения изменнеий в проект.

     \item Вывод: добавление других участников в проект работает корректно.

    \end{enumerate}

\item Добавление других участников в проект:
    \begin{enumerate}
    \item Сценарий: вставка корректного DBML-кода с определением двух таблиц и связи между ними.
    
    \item Результат: код успешно парсится, в визуальном редакторе автоматически отображаются соответствующие таблицы и связи, синтаксис подсвечивается.
    
    \item Сценарий: ввод некорректного DBML-синтаксиса.
    
    \item Результат: проблемные строки выделяются красным, выводится сообщение с описанием ошибки, кнопка сохранения неактивна.
    
    \item Вывод: редактор обеспечивает полный цикл работы с DBML-кодом.

    \end{enumerate}

\item Работа интерфейса диаграмм:
    \begin{enumerate}
    \item Сценарий: создание двух таблиц через графический интерфейс.
    
    \item Результат: таблицы отображаются на холсте с возможностью перемещения, сохраняются пропорции и читаемость текста.
    
    \item Сценарий: установка связи типа "один-ко-многим" между таблицами.
    
    \item Результат: связь отображается соответствующей стрелкой, при перемещении таблиц соединение сохраняется.
    
    \item Сценарий: обновление страницы после изменений.
    
    \item Результат: все элементы диаграммы восстанавливаются в прежних позициях, связи сохраняются.
    
    \item Вывод: визуальный редактор обеспечивает стабильную работу со схемой БД.

    \end{enumerate}
    
\end{enumerate}

\textbf{3.3 Общие выводы по результатам тестирования}
\addcontentsline{toc}{subsection}{3.3 Общие выводы по результатам тестирования}

Общие выводы по результатам функционального тестирования
показывают, что разработанное приложение полностью соответствует
заявленным требованиям и демонстрирует высокую стабильность работы. Все
ключевые функции — регистрация
и авторизация, выход из системы, создание проекта, удаление проекта, добавление в избранное, добавление других участников в проект, работа dbml редактора, работа интерфейса для создания диаграмм, сохранение проекта, экспорт SQL-дампа — были тщательно проверены и
функционируют корректно.

В процессе тестирования не было выявлено критических ошибок,
приводящих к сбоям или потере данных. Приложение устойчиво к
некорректным действиям пользователя: при вводе неверных данных, повреждении базы данных или попытке работы с
неподдерживаемыми форматами всегда выводятся информативные
сообщения об ошибках, а интерфейс остаётся отзывчивым. Все сценарии,
связанные с обработкой ошибок, реализованы на высоком уровне, что
обеспечивает положительный пользовательский опыт даже в нестандартных
ситуациях.

Особое внимание уделялось производительности и удобству
интерфейса. Приложение быстро реагирует на действия пользователя,
мгновенно обновляет список проектов фильтров и страницу редактирования проекта, быстро работает синхронизация между диаграммой и dbml.
Интерфейс интуитивно понятен, все элементы управления логично
расположены, а переходы между окнами происходят без задержек и с
сохранением состояния.

В целом, результаты тестирования свидетельствуют о высокой
надёжности, удобстве и безопасности приложения. Продукт готов к
внедрению, дальнейшему использованию и масштабированию, а также может
быть рекомендован для распространения среди широкой аудитории
пользователей.

\

\textbf{3.4  Стратегия продвижения веб-приложения}
\addcontentsline{toc}{subsection}{3.4  Стратегия продвижения веб-приложения}

Для успешного вывода приложения на рынок и привлечения
пользователей необходима продуманная стратегия продвижения, сочетающая
современные digital-инструменты, работу с сообществом и создание
качественного контента.

Позиционирование продукта: приложение подаётся как
удобный инструмент для проектирования БД с двусторонней синхронизацией DBML и диаграмм, а также с коллаборативная платформа для командной работы над схемами. Веб0приложение представляет собой образовательное решение для изучения основ проектирования баз данных.

Основной акцент делается на digital-маркетинг. Создаётся официальный
сайт или лендинг с подробным описанием возможностей, скриншотами,
видео-демонстрациями и формой для скачивания. Для повышения доверия
публикуются отзывы первых пользователей и кейсы применения.

Активно используются социальные сети (VK, Telegram, YouTube) для публикации новостей, обучающих
роликов, советов по использованию и анонсов обновлений. Ведётся блог или
раздел с инструкциями и ответами на частые вопросы. Для быстрого старта
запускается таргетированная реклама в соцсетях и поисковых системах по
релевантным запросам (например, «онлайн конструктор схем БД», «DBML редактор», «визуальное проектирование SQL»).

Планируется сотрудничество с тематическими сообществами и
форумами: публикация анонсов, участие в обсуждениях, ответы на вопросы
пользователей.

Особое внимание уделяется работе с обратной связью: оперативно
обрабатываются отзывы, предложения и жалобы, что позволяет быстро
улучшать продукт и формировать лояльное сообщество. Для поддержки
пользователей создаётся отдельный канал связи (email, Telegram-бот, форма
на сайте).

Для выхода на профессиональный рынок и образовательный сектор
приложение предлагается школам, вузам, компаниям и библиотекам с
возможностью кастомизации и интеграции. Участвуя в профильных
выставках, конкурсах и хакатонах, продукт получает дополнительное
признание и охват.

В долгосрочной перспективе стратегия включает регулярные
обновления, расширение функционала, публикацию новых обучающих
материалов и развитие партнерских программ. При необходимости
подключается PR-поддержка: публикации в IT-СМИ, интервью с
разработчиками, обзоры у блогеров и лидеров мнений.
Таким образом, стратегия продвижения строится на сочетании
качественного продукта, активной работы с аудиторией, грамотного контентмаркетинга и постоянного совершенствования сервиса.

\

\textbf{3.5 Направления развития приложения}
\addcontentsline{toc}{subsection}{3.5 Направления развития приложения}

Направления развития веб-приложения для визуального проектирования баз данных включают расширение функциональных возможностей, улучшение совместной работы, интеграцию с внешними системами, добавление образовательных функций, повышение удобства использования и обеспечение безопасности.

Одним из ключевых направлений развития является поддержка дополнительных типов баз данных, включая NoSQL-системы, такие как MongoDB, Neo4j и Cassandra. Важно реализовать визуализацию для документоориентированных, графовых и колоночных СУБД, а также автоматическую генерацию схем, учитывающую особенности их структур, например коллекции, документы, узлы и связи. Дополнительно необходимо поддерживать облачные базы данных, включая Firebase, AWS DynamoDB и Google Firestore, с возможностью экспорта схем в форматы, совместимые с облачными провайдерами.

Другим направлением является развитие функционала DBML-редактора, включая автодополнение, интеграцию с Language Server Protocol для подсказок по синтаксису, валидацию кода в реальном времени и поддержку быстрого рефакторинга, например, переименование таблиц и полей. Также важно добавить шаблоны типовых решений — готовые схемы для e-commerce, социальных сетей и CRM-систем с возможностью создания пользовательских шаблонов.

Необходимо внедрить инструменты анализа и оптимизации, включая автоматическую нормализацию для выявления избыточности данных и рекомендации по декомпозиции таблиц, а также проверки производительности схем с симуляцией нагрузочного тестирования и анализом потенциальных узких мест.

Для улучшения совместной работы требуется поддержка режима реального времени с возможностью одновременного редактирования схемы несколькими пользователями и визуализацией их действий, включая отображение курсоров и выделений. Следует реализовать систему комментариев и обсуждений, чтобы пользователи могли оставлять заметки к элементам схемы и получать уведомления об изменениях. Также важно добавить полноценное управление версиями с возможностью ветвления схем, сравнения изменений, отката к предыдущим состояниям и ведения журнала активности с указанием авторов правок.

Интеграция с внешними системами должна включать генерацию миграций в SQL-скрипты для популярных фреймворков, таких как Django, Laravel и Rails, а также поддержку инструментов типа Flyway и Liquibase. Важно обеспечить возможность использования CI/CD-процессов, включая автоматическую проверку схем в pipeline и webhooks для синхронизации с репозиториями.

Отдельное внимание стоит уделить образовательным функциям, включая пошаговые руководства по проектированию баз данных, практические задания с автоматической проверкой и визуализацию выполнения SQL-запросов на схеме с анимацией операций JOIN. Для поддержки учебного процесса важно реализовать режим групповых проектов с ролевой моделью для преподавателей и студентов, а также систему оценки работ. Экспорт в учебные форматы, такие как PDF, PPTX и интерактивные HTML-демонстрации, позволит адаптировать платформу для образовательных учреждений.

Улучшение пользовательского опыта включает возможность персонализации интерфейса, например выбор тем оформления, цветовых схем, гибкую настройку расположения панелей и поддержку доступности, включая работу с экранными читалками и режим высокой контрастности. Для повышения производительности следует оптимизировать рендеринг крупных схем с более чем сотней таблиц, реализовать ленивую загрузку элементов и предусмотреть оффлайн-режим с локальным сохранением проектов и синхронизацией при восстановлении соединения.

Дополнительно важно реализовать автоматическое резервное копирование данных с ежедневным сохранением в облачное хранилище и возможностью восстановления из точки сохранения.

Представленный план развития охватывает технические, функциональные и пользовательские аспекты приложения. Основные приоритеты — это гибкость, позволяющая поддерживать разные СУБД и интеграции, коллаборация с инструментами для командной работы, образовательные возможности с встроенными обучающими ресурсами и масштабируемость для корпоративного использования. Реализация этих направлений создаст универсальную платформу для проектирования баз данных, подходящую как для разработчиков, так и для образовательных учреждений и бизнеса.

\

\textbf{3.6 Выводы}
\addcontentsline{toc}{subsection}{3.6 Выводы}

В результате проведённого функционального тестирования приложение
продемонстрировало высокую стабильность, надёжность и соответствие
заявленным требованиям. Все ключевые функции — регистрация
и авторизация, выход из системы, создание проекта, удаление проекта, добавление в избранное, добавление других участников в проект, работа dbml редактора, работа интерфейса для создания диаграмм, сохранение проекта, экспорт SQL-дампа — работают корректно и
обеспечивают положительный пользовательский опыт. Приложение
устойчиво к некорректным действиям пользователя и быстро реагирует на
изменения, что подтверждает его готовность к внедрению и дальнейшему использованию.

Стратегия продвижения приложения строится на сочетании
качественного digital-маркетинга, активной работы с сообществом,
публикации обучающих материалов и обратной связи с пользователями.
Использование социальных сетей, тематических форумов, сотрудничество с
блогерами и участие в профильных мероприятиях позволяет быстро донести
информацию о продукте до целевой аудитории, сформировать лояльное
сообщество и обеспечить стабильный рост пользовательской базы.

Вектор развития приложения включает поддержку разных СУБД и интеграций, коллаборации с инструментами для командной работы, образовательные возможности с встроенными обучающими ресурсами и масштабируемость для корпоративного использования. В перспективе возможно внедрение
технологий искусственного интеллекта для генерации dbml кода и создании диаграммы.

Таким образом, приложение обладает прочной технической основой,
востребованным функционалом и широкими возможностями для
масштабирования и развития, что делает его перспективным продуктом для
дальнейшего продвижения и коммерческого успеха.